\documentclass[14pt]{extreport}
\usepackage[a5paper, landscape, margin=1cm]{geometry}
\usepackage[czech]{babel}
\usepackage{lettrine}
\usepackage[final]{microtype}
\usepackage{hyperref}
\usepackage[svgnames]{xcolor}
\usepackage{fancyhdr}
\usepackage{lettrine}
\usepackage[final]{microtype}
\usepackage[autostyle]{csquotes}
\usepackage[natbib=true]{biblatex}
\usepackage{url}
\usepackage{svg}


\renewcommand*{\LettrineTextFont}{\normalfont}

\pagenumbering{gobble}

\begin{document}
\begin{center}
    \includesvg[width=5em, angle=90]{leaf.svg}
\end{center}

\vspace*{-16em}

\subsection*{Odraz přírody}
\lettrine[lines=3, findent=5pt, nindent=0pt]{K}{dyž} vstoupíte do této městské džungle, mějte oči dokořán, protože 
každá dlažební kostka tu může nést tajnou zprávu. V deseti záběrech se přistavujeme u malého zrcadla — ona je to spíš 
pohlednice přírody zaslaná z vnějšího světa, kde slunce hraje prim a listy šeptají příběhy větru.

Zrcátko tu nezastupuje úřední razítko spásy, ale je to dvorní šašek, co si dělá legraci z šedivých staveb a kovových 
zábran. Nečekaně vpustí do rámu živou zeleň: větvičku od vedlejšího stromu, list omotaný pavučinou nebo kousek 
trávy, který se vydal na průzkum chodníku.

\vspace{2em}

Tyto fotografie nevyprávějí hrdinské eposy, ani neposílají monumentální salut. Stačí jim ten drobný dialog mezi městem 
a přírodou.  I přesto, že 
betonový ráj přehluší šepot listí a buduje zábrany, přírodu stále lze spatřit.

\vspace{3em}

\mbox{}\hfill{18. 5. 2025}

\mbox{}\hfill{Jakub Adamec}

\end{document}