\documentclass[12pt]{article}

\usepackage[czech]{babel}
\usepackage[a4paper, left=3cm, right=3cm, top=3cm, bottom=3cm]{geometry}
\usepackage[svgnames]{xcolor}
\usepackage{fancyhdr}
\usepackage[final]{microtype}
\usepackage[autostyle]{csquotes}

\definecolor{DarkGray}{gray}{0.3}

\begin{document}
\begin{center}
    \Huge\bfseries\color{DarkGray} {Město v zrcadle} \\
    \Large \textbf{Kontrast umělého a živého} \\ 
    \normalsize Koncept projektu, A7B33DIF \\
    Jakub Adamec
\end{center}

V sérii fotografií se chci zaměřit na nečekané setkání dvou světů --- strohé, geometrické struktury městského prostředí 
a organickou krásu přírody. Do každého záběru umísťuji malé zrcátko, které zrcadlí skutečný kus přírody (strom, listí) 
tak, aby byl v kontrastu s betonovými stěnami, kovovými zábradlími, popelnicemi či městskými značkami. Zrcadla 
reflektují věrnou (needitovanou) realitu, čímž vzniká paralelní \enquote{okno do světa}, který se jinak v obraze 
neobjeví.

\subsection*{Cíle projektu}
\begin{itemize}
    \item Kontrast dvou realit: vizuálně a konceptuálně propojit město (umělé, statické) s přírodou (živou, dynamickou).
    \item Svět v odrazech: ukázat, jak lze malým zrcadlem změnit celkovou atmosféru a význam scény.
    \item Autenticita záběrů: zrcadlené prvky jsou vždy čerpány z bezprostředního okolí a bez digitální manipulace.
    \item Fúze umění a konceptu: nabídnout divákovi prostor k zamyšlení nad vztahem člověka k přírodě i nad tím, co vše 
    nám město „kazí“ či skrývá.
\end{itemize}

Projekt nabízí pohled na to, jak malý fragment přírody může prorazit betonovou šedi a přimět nás znovu vnímat okolní 
svět.

\end{document}