\documentclass[11pt,a4paper]{article}
\usepackage[czech]{babel}
\usepackage[margin=0.85in]{geometry}
\usepackage{graphicx}

\usepackage{cfg}

\author{Jakub Adamec}

\begin{document}
\begin{center}
    \Huge Jakub Adamec \\
    \Large 12\_barvy, A7B33DIF \\ 
\end{center}

\section{Otázka 3} Jakou barvu (vyjádřete v CIE XYZ souřadnicích) reprezentuje hodnota RGB \enquote{100, 180, 70}
vzhledem k prostoru Demo RGB, jehož primární barvy mají CIE XYZ souřadnice
\begin{align}
    D_r = \begin{bmatrix} 0.4124 \\ 0.2127 \\ 0.0193 \end{bmatrix}\text{, } \, 
    D_g = \begin{bmatrix} 0.3576 \\ 0.7152 \\ 0.1192 \end{bmatrix}\text{, } \,
    D_b = \begin{bmatrix} 0.1805 \\ 0.0722 \\ 0.9504 \end{bmatrix}\text{,}
\end{align}
a hodnota gamma je rovna jedné (jde o lineární prostor)?

1. normalisujeme RGB hodnoty

\begin{align}
    R=100, G=180, B=70, \, \, \text{tedy } C = \begin{bmatrix}R \\ G \\ B \end{bmatrix}_{\text{norm}}
    = \frac{1}{255}\begin{bmatrix}100 \\ 180 \\ 70 \end{bmatrix} = 
    \begin{bmatrix} 0.3922 \\ 0.7059 \\ 0.2745\end{bmatrix}
\end{align}

2. výpočet vektoru CIE XYZ souřadnic barvy získané ve barevném prostoru Demo RGB
\begin{align}
    C_D = 0.3922 D_r + 0.7059 D_g + 0.2745 D_b = 
    \begin{bmatrix}
        0.4124 & 0.3576 & 0.1805 \\ 0.2127 & 0.7152 & 0.0722 \\ 0.0193 & 0.1192 & 0.9504
    \end{bmatrix}
    \begin{bmatrix} 0.3922 \\ 0.7059 \\ 0.2745 \end{bmatrix} = \begin{bmatrix} 0.4637 \\ 0.6081 \\ 0.3526 \end{bmatrix}
\end{align}

\section{Otázka 4} Jaké RGB hodnoty je třeba poslat do monitoru Supersvit a Ostrovid, aby zobrazily barvu, jejíž XYZ
souřadnice jste vypočítali v odpovědi na otázku 3?
\[
    C_D = \begin{bmatrix} 0.4637 \\ 0.6081 \\ 0.3526 \end{bmatrix}
\]
Musíme vyřešit rovnici 
\begin{align}
    C_D = M \begin{bmatrix} R \\ G \\ B \end{bmatrix}\text{,}
\end{align}
kde $M$ je matice primárních barev (pro Supersvit nebo Ostrovid).

Matice primárních barev pro Supersvit:
\[
    M_S = 
    \begin{bmatrix}
        0.4065 & 0.3191 & 0.1684 \\ 0.2127 & 0.7152 & 0.0722 \\ 0.0063 & 0.0660 & 0.9625 \\
    \end{bmatrix}
\]
Matice primárních barev pro Ostrovid:
\[
    M_O = 
    \begin{bmatrix}
        0.3815 & 0.3901 & 0.1714 \\ 0.2127 & 0.7152 & 0.0722 \\ 0.0313 & 0.1950 & 0.6587 \\
    \end{bmatrix}
\]
\subsection{Supersvit}
\begin{align}
    A = \begin{bmatrix} R_S \\ G_S \\ B_S \end{bmatrix} = M_S^{-1} C_{D} = 
    \begin{bmatrix} 0.4790 \\ 0.6758 \\ 0.3169 \end{bmatrix}
\end{align}

\subsection{Ostrovid}
\begin{align}
    B = \begin{bmatrix} R_O \\ G_O \\ B_O \end{bmatrix} = M_O^{-1} C_{D} = 
    \begin{bmatrix} 0.3448 \\ 0.7167 \\ 0.3067 \end{bmatrix}
\end{align}

\vspace{4em}
Všechny výpočty byly provedeny v jazyce 
\raisebox{-0.2em}{\href{https://julialang.org}{\includegraphics[height=1em]{julia-logo.pdf}}}.

\end{document}