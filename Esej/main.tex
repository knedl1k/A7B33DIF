\documentclass[12pt]{article}

\usepackage[czech]{babel}
\usepackage[a4paper, left=3cm, right=3cm, top=3cm, bottom=3cm]{geometry}
\usepackage{hyperref}
\usepackage[svgnames]{xcolor}
\usepackage{fancyhdr}
\usepackage{lettrine}
\usepackage[final]{microtype}
\usepackage[autostyle]{csquotes}
\usepackage[natbib=true]{biblatex}
\usepackage{url}
\setcounter{biburllcpenalty}{7000}
\setcounter{biburlucpenalty}{8000}

\addbibresource{references.bib}

\definecolor{DarkGray}{gray}{0.3}

\babelprovide[transforms = oneletter.nobreak]{czech} 

\usepackage{titlesec}
\titleformat{\section}
  {\normalfont\Large\bfseries\color{DarkGray}}
  {\thesection}
  {1em}
  {}
\titleformat{\subsection}
  {\normalfont\large\color{DarkGray}}
  {\thesubsection}
  {1em}
  {}

\pagestyle{fancy}
\fancyhf{}
\fancyfoot[C]{\thepage}
\renewcommand{\headrulewidth}{0pt}
\renewcommand{\footrulewidth}{0pt}
\renewcommand*{\LettrineTextFont}{\normalfont}
\setlength{\headheight}{14.5pt}

\hypersetup{
    colorlinks=true,
    linkcolor=DarkGray,
    citecolor=DarkGray,
    urlcolor=DarkGray
}

\title{%
  Obrázek si uděláš, \\
  \large když ti ho správně ukážou}
\author{Jakub Adamec}
\date{květen 2025}

\begin{document}

\makeatletter
\begin{titlepage}
    \centering
    \vspace*{1cm}

    {\Huge\bfseries \textcolor{DarkGray}{\@title} \par}
    \vspace{2.5cm}

    {\Large\bfseries \textcolor{DarkGray}{\@author} \par}
    \vspace{0.2cm}

    {\bfseries \textcolor{DarkGray}{adamecjakub@icloud.com} \par}
    \vspace{1.5cm}

    {\large\bfseries \textcolor{DarkGray}{\@date} \par}
    \vfill

    \vspace*{1cm}
    {\large Zpracování digitální fotografie \\ Katedra kybernetiky \\ ČVUT FEL\par}
\end{titlepage}
\makeatother

\pagestyle{fancy}

\section{Úvod}
\lettrine[lines=3, findent=5pt, nindent=0pt]{V}{isuální} Babylón --- tam, kde nás každé ráno vítá 
armáda mžourajících čtverečků a blikajících obláčků, které vtrhují do našich očí miliardy obrazů. Dříve se fotografii 
přisuzovala jakási alchymie světla a stínu, anebo hříšné čertoviny na fotopapíru; dnes však cvaknutí drobné šmouhy mění 
svět k nepoznání a stává se hlavním lajtmotivem\footnote{z něm. Leitmotiv, \enquote{vodící motiv}} našeho společného 
života.

Masivní konzumace obrázků nám na jedné straně otevírá brány svobodomyslné kreace --- dává dorazit všem maličkým i 
velkým umělcům, kteří by jinak zůstali schovaní za zdmi svých ateliérů. Na straně druhé ale v tom šíleném švihu narůstá 
i rámus a chaos: pozornost se štěpí jak staré sklo na tisíce úlomků, emoce praskají jak cirkusová prskavka a každý 
záblesk radosti či strachu tančí s námi jen chvilku, než ho přebije další roztodivná fotka z cizího města, kavárny či 
neuspořádané snídaně.

V tom všem ale zapomínáme, nebo snad záměrně přehlížíme, že umění, a tedy i fotografie, slouží jako nástroj propagandy 
od nepaměti. Vývojem technologií a zdokonalováním jednotlivých procesů člověk pomalu neví, jestli to, co vidí, 
je svět, nebo jen nějaká cizí představa o něm.

\section{Jak se stane, že nemůžeme věřit fotografii?} 
\subsection{Kaskáda náhody a scenérie {náhodně} nastražená}
Fotograf si hraje na náhodu, jako by hodil šipku se zavázanýma očima. \enquote{Já jen cvaknul,} hlásí, ale ví moc dobře, 
že přivstát si k růžovému úsvitu je skoro jako objednat si lístek do ráje. Však ten pohled, kdy růžové mraky hladí 
obzor, roztaje srdce i ledovou sochu.

Vybavme si třeba starou lavičku u kříže. Divák v tom snad cítí dávnou tragédii, nostalgii, dohru starého příběhu --- 
a přitom to bývala jen zaprášená zastávka na lesní pěšině. Stačí pár slov v popisku a z prázdného místa děláš scénu se 
srdceryvným dramatem.

\subsection{Barvy jako politické heslo}
Kdo by nemiloval teplé slunce? Ale když máš rudou vlajku za zády, už to není romantika, ale revoluce. Nebo naopak: 
matný modrý filtr a hned z armády uděláš tiché básníky klidu. Když samotné barvy začnou mluvit, myšlenky se řadí do řady 
podle vloženého scénáře, i kdyby ti prodávali obnošené ponožky.

\subsection{Digitální divadlo s umělými herečkami}
Ulice plná davu, tanky mezi stromy, plameny kdesi za rohem --- a přitom nikde nic. Stačí špetka umu a prázdné parky 
ožijí jako kulisy na jevišti lži. \enquote{Kde se to stalo?} ptáš se: \enquote{Tady i tady,} odpoví ti algoritmus s 
nekonečnou fantazií.

Program načaruje zbraně i barikády, a najednou se z nevinného protestu stane občanská válka v HD rozlišení. Všechny ty 
příběhy vytvořené z jedniček a nul vypadají tak pravě, že zlákají i nejotrlejšího skeptika.

\section{Kdo drží spoušť, drží pravdu}
\subsection{Fotografie jako nástroj konstrukce národní identity}
V době, kdy Spojené státy postrádaly tisíciletou historii evropských národů, svobodní Američané nelenili, vzali foťák a 
vytvořili si vlastní. Carleton Watkins vylepil mamutí desky 
Yosemit\footnote{\url{https://carletonwatkins.org}} --- skalní věže jako gotické katedrály, mlžný opar jak biblické 
vidění --- a zařídil, že stát uctí divočinu tím, že ji prohlásil posvátnou. Manifest Destiny? Spíš manifest fotografie. 
Jeho obrazy se v žurnálech a poštovních pohlednicích šířily dál než dopisy, a když je spatřili lidé v New Yorku či 
Paříži, uvěřili, že USA stojí na práhu nového Edenu. Watkinsova fotoexpedice nadhodila otázku: co je skutečné --- 
krajina, nebo to, jak ji zobrazíme?

Buffalo Bill Cody poslal do světa obrázky, kde kavalérie v modrém \enquote{civilizovala} rudé plameny indiánských vozů, 
a všichni tloukli pěstmi pro smíření --- přitom maskovali genocidu. Jeho plakát 
\enquote{Enemies in '76, Friends in '85}\cite{buffalo} nebyl reklama, ale obrazová pohádka, která do Chicaga přetáhla 
dav větší než na průmyslové vynálezy. Cesty parníkem a železnicí rozvezly tyto obrazy po Evropě, kde se Ameriku učili 
milovat. Bez informací o skutečných cenách, které domorodí obyvatelé zaplatili. Každý kontrast rudé a modré byl pečlivě 
komponován jako divadelní scéna, kde všichni herci znali své role, jen divák ne.

\subsection{Sovětské čistky na filmovém pásu}
Fotomontáž LeFu\footnote{\url{https://encyklopedie.soc.cas.cz/w/LeF}} začala jako avantgarda, skončila jako kyberspirála 
zapomnění. Trockij zmizel z Leninovy strany, pak z fotky, až zbyl sólista před anonymním davem \cite{trockij}. V roce 
1938 na tom dělala síť retušérů na plný úvazek --- to je vášeň pro čistotu dějin. Retušér dokázal zaměnit i architekturu 
budov v pozadí, přidat orámování nebe, aby působilo monumentálně, či ubourat sochu, která se politicky nehodila. 
Historie se tak psala na koleni s brkem Photoshopu 20. let.\cite{retus}

Chaldejův snímek vztyčení vlajky\cite{reichstag}, bez hodinek, více dýmu, zřetelnější stíny, se stal kronikou 
stalinských priorit. Každá retuš byla výpovědí víc než tisíc slov z úst oficiálního prohlášení. V historických 
učebnicích se tato fotografie objevuje s poznámkou o \enquote{optimální kompozici}, jako by šlo o umělecký záměr, ne o 
politický kalkul. I dnes se historici dohaduji, která verze je \enquote{pravější}, a tak se z politického triku stala 
studnice živé vědy.
\newpage
\subsection{Umělá realita}
V éře chytrých mobilů se fotografická montáž stala hlavní zbraní. Pravice edituje kampaňové snímky kandidátů tak, 
aby stáli před vlajkami a symboly, které rezonují s konzervativním publikem; levice zase sestavuje koláže protestů 
a sociálních statistik, aby vypadalo, že masy burcují k revoluci. Každá vrstva montáže, od barevného filtru až po 
skryté vodoznaky, přidává k obrazu ideologické váhy, a video-sestřih GIFů přetváří nevinnou fotografii v manifest.

Fotografie už neukazuje to, co bylo. Ukazuje to, co by bylo, kdybychom chtěli věřit pohádce.
A výsledek? I když fotografie může vypadat nefalšovaně, je napůl snem, napůl plánem iluzionisty v pozadí.

\section{Existuje obrana?}
Asi každému je povědomý ten pocit: spatřím fotku, a hned věřím, že svět je takový, jak mi ho servírují. Ale než začneme 
mlčky souhlasit, ptejme se: \enquote{Kdo za tím cvaknutím stojí? Co tím sleduje a proč to zveřejnil?} Je to ten pán s 
fotoaparátem, co si potřebuje přihřát polívčičku, nebo umělec s duší malíře, co chce ukázat pravdu? Nebo se jen baví 
tím, jak nás obalamutit?

Vizuální gramotnost není žádná čarodějnice, ale cvik jako v tělocvičně. Měli bychom se naučit číst kompozici jako 
noviny: kde je světlo, kde stín a proč zrovna tady ta postava kouká jinam než ostatní.

\section{Závěr}
\lettrine[lines=3, findent=5pt, nindent=0pt]{N}{e} ve všem je propaganda, podsouvání cizího pohledu na svět. Ne každý má 
potřebu druhým říkat, co smějí a nesmějí. Ale zároveň bychom neměli sami podlehnout pohodlnosti a myslit si, že můžeme 
ignorovat všudypřítomné pokusy vlivných i nevlivných o změnu, či snad jen pokoušení našeho pohledu na věc.

Kritické myšlení není jen žonglování s filozofickými pojmy, ale také schopnost zahlédnout skryté klamy. Když fotka volá 
po emocích, ptáš se: \enquote{Proč mi zvedá adrenalin nebo slzy? Co je v pozadí, čeho si nevšimnu?} A pak ještě: 
\enquote{Jaké mám já předsudky, když na to celé nadhodím svůj názor?}

\vspace{6em}

\begin{center}
    Fotografie je hezká věc, ale jakmile ji někdo podepíše, můžeme si být jistí, \\
    že už neříká pravdu, ale svůj dobrý úmysl.
\end{center}

\newpage
\section{Podklady}
\nocite{*}
\printbibliography[heading=none,title={}]

\end{document}