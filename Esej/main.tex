\documentclass[12pt, a4paper]{article} % Nebo scrartcl pro více možností

\usepackage[T1]{fontenc}
\usepackage{fontspec}
\usepackage[czech]{babel}
\usepackage{graphicx}
\usepackage[a4paper, left=3cm, right=3cm, top=3cm, bottom=3cm]{geometry} % Větší okraje
\usepackage{amsmath}
\usepackage{hyperref}

\usepackage[svgnames]{xcolor}
\usepackage{fancyhdr}
\usepackage{lettrine}
\usepackage{microtype}
\usepackage[autostyle]{csquotes} % smart quote handling
% \usepackage{ebgaramond}

\definecolor{DarkGray}{gray}{0.3}
\definecolor{LightGray}{gray}{0.8}

\babelprovide[transforms = oneletter.nobreak]{czech} 

\usepackage{titlesec}
\titleformat{\section}
  {\normalfont\Large\bfseries\color{DarkGray}}
  {\thesection}
  {1em}
  {}
\titleformat{\subsection}
  {\normalfont\large\itshape\color{DarkGray}}
  {\thesubsection}
  {1em}
  {}

\pagestyle{fancy}
\fancyhf{}
\fancyfoot[C]{\thepage}
\renewcommand{\headrulewidth}{0pt}
\renewcommand{\footrulewidth}{0pt}
\renewcommand*{\LettrineTextFont}{\normalfont}
\setlength{\headheight}{14.5pt}

\hypersetup{
    colorlinks=true,
    linkcolor=DarkGray,
    citecolor=DarkGray,
    urlcolor=DarkGray
}

\title{%
  Visuální babylón \\
  \large aneb Masová konzumace fotografií}
\author{Jakub Adamec}
\date{květen 2025}

\begin{document}

\makeatletter
\begin{titlepage}
    \centering
    \vspace*{1cm}

    {\Huge\bfseries \textcolor{DarkGray}{\@title} \par}
    \vspace{0.5cm}

    {\Large \@author \par}
    \vspace{0.2cm}

    {knedl$1$k@tuta.io \par}
    \vspace{1.5cm}

    {\large \@date \par}
    \vfill

    % \includegraphics[width=0.9\textwidth, height=0.4\textheight, keepaspectratio]{titulni_obrazek.jpg}

    \vspace*{1cm}
    {\large Zpracování digitální fotografie \\ Katedra kybernetiky \\ ČVUT FEL\par} % Informace o škole/katedře
\end{titlepage}
\makeatother

\pagestyle{fancy}

\section{Úvod}
\lettrine[lines=3, findent=5pt, nindent=0pt]{V}{isuální} Babylón — tam, kde nás každé ráno vítá 
armáda mžourajících čtverečků a blikajících obláčků, vtrhující do našich očí miliardy obrazů v čase tak rychlém, že by 
se i Jára Cimrman nestačil divit. Dříve se fotografii přisuzovala jakási alchymie světla a stínu, anebo 
hříšné čertoviny na fotopapíru; dnes však cvaknutí drobné šmouhy mění svět k nepoznání a stává se hlavním 
lajtmotivem\footnote{z něm. Leitmotiv, \enquote{vodící motiv}} našeho společného života.

Ta masivní konzumace obrázků nám na jedné straně otevírá brány svobodomyslné kreace — dává dorazit všem maličkým i 
velkým umělcům, kteří by jinak zůstali schovaní za zdmi svých ateliérů. Na straně druhé ale v tom šíleném švihu narůstá 
i rámus a chaos: pozornost se štěpí jak staré sklo na tisíce úlomků, emoce praskají jak cirkusová prskavka a každý 
záblesk radosti či strachu tančí s námi jen chvilku, než ho přebije další roztodivná fotka z cizího města, kavárny či 
neuspořádané snídaně na Instagramu.

\section{}

\end{document}