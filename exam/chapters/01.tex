\section{První týden}

\subsection{Jaké jsou rozdíly mezi analýzou obrazu (počítačovým viděním) na jedné straně a počítačovou grafikou na druhé 
straně? Uveďte dva příklady, které rozdíly demonstrují.}

Analýza obrazu (počítačové vidění) a počítačová grafika jsou dvě komplementární, ale zásadně odlišné disciplíny. Zatímco 
počítačové vidění se soustředí na interpretaci a extrakci informací z reálných snímků, počítačová grafika se zaměřuje na 
syntézu a generování obrazů z modelů či algoritmů. Oba přístupy jsou dnes stále více propojeny a společně vytvářejí nové 
možnosti pro budoucí aplikace.

Příklad 1: Autonomní vozidla
\begin{itemize}
    \item Počítačové vidění: Kamery a senzory na vozidle zachycují reálné snímky, které jsou analyzovány pomocí 
    algoritmů pro detekci chodců, vozidel či dopravních značek.
    \item Počítačová grafika: Pro trénink a simulaci autonomních systémů se vytvářejí virtuální prostředí, kde jsou 
    realistické scénáře generovány počítačem. Tato simulace umožňuje testování a optimalizaci algoritmů v bezpečném, 
    kontrolovaném prostředí.
\end{itemize}


Příklad 2: Zdravotnictví
\begin{itemize}
    \item Počítačové vidění: Analýza lékařských snímků (rentgenů, MRI či CT) umožňuje automatickou detekci abnormalit, 
    jako jsou nádory nebo jiné patologické změny, což napomáhá přesnější a rychlejší diagnostice.
    \item Počítačová grafika: Vytváření 3D modelů vnitřních orgánů z dat získaných z lékařských snímků pomáhá chirurgům 
    lépe plánovat operace a vizualizovat složité anatomické struktury.
\end{itemize}

\subsection{}